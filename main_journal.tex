\documentclass[a4paper,num-refs,strutturaTemplate]{style}
\journal{strutturaTemplate}
\usepackage[italian]{babel}

\usepackage{amsmath,amssymb}

\usepackage{graphicx}
\usepackage{subfig}		% requires caption
%\usepackage{tikz}
%\usetikzlibrary{patterns,plotmarks}

\usepackage{siunitx}
\sisetup{
separate-uncertainty,
separate-uncertainty-units = bracket}


\usepackage{floatrow}
\floatsetup[table]{style=plaintop} % altrim mette le caption dle table in basso (nonostante il setup di caption)
\newfloatcommand{capbtabbox}{table}[][\FBwidth] % Table float box with bottom caption, box width adjusted to content
\newfloatcommand{ffiggbox}{figure}[][\FBwidth] % Figure float box with bottom caption, box width adjusted to content

\usepackage{import} 
\usepackage{comment}

\title{Titolong}

\author[1]{Jacopo Lancione}
\author[1]{co.}

\affil[1]{affiliazione}

\begin{document}
\begin{frontmatter}
\maketitle
\thispagestyle{empty}
%\begin{abstract}
% chiacchiere
%\end{abstract}
\tableofcontents
\end{frontmatter}

%\newpage

\begin{comment}
%%%%%%%%% CHECK da FARE %%%%%%%%%
- um
- cifre significative
- c'è tutto di tutti i FIT? etichette degli assi, caption, pvalue, parametri (quelli non citati/inutili solo nel riquadrino nell'immagine del fit o in tabella in appendice)
- tutte le REF portano nel posto giusto?
- TABELLE. Ci va bene come e dove sono messe? Vogliamo uniformare tra fwhm e sigma? sigma
- inserire riferimenti al NIST e per Crystalball (secondo me basta una footnote) sì mettile
- sono chiare le procedure di misura?
- rileggere la scheda di laboratorio per essere sicuri di non aver perso per strada nulla
- ci piacciono le osservazioni che abbiamo fatto (ale ha fatto dei commenti sugli errori e su come migliorare la misura, vogliamo mettere cose simili anche nelle altre sezioni?)
- sbarazzarsi delle parti ripetute


\end{comment}

%\import{path}{file_name}


\begin{comment}
 \newpage
{
\appendix
\import{}{appendice}
} 
\end{comment}

\end{document}

%\cite{knuth:1984}
%\bibliography{_bib/paper-refs}
