%  ____   __   ___   _  _   ____   ____ 
% (  __) (  ) / __) / )( \ (  _ \ (  __)
%  ) _)   )( ( (_ \ ) \/ (  )   /  ) _) 
% (__)   (__) \___/ \____/ (__\_) (____)
 
- Basta copiare il codice e inserire nome dell'immagine e caption

- Se nel testo vogliamo richiamare all'attenzione una figura bisogna usare \ref{fig:example_label} e nel codice della figura togliere il commento dal comando \label{fig:example_label} sostituendo "example_label" con qualcosa di evocativo…
(NN "figura1" per pietà!! altrimenti impazziamo)
Se non ce ne frega ignoriamo \label e lasciamolo commentato

- Per cambiare le dimensioni modificare .45 che sarebbe la larghezza dell'immagine espressa come frazione della larghezza della pagina (ma anche di questo possiamo pensarci alla fine)

- Il consiglio sportivo è di sbattersene di dove l'immagine viene messa, ci pensiamo alla fine (quindi lasciamo htbp)




% Per inserire UNA FIGURA
\begin{figure}[htbp]
   \centering
   \includegraphics[width=.45\textwidth]{file_name} % da modificare
   \caption{Example caption} % da modificare
%   \label{fig:example_label}
\end{figure}



% Per affiancare DUE FIGURE, si può mettere una didascalia/caption globale
\begin{figure}[htbp]
\centering
	\subfloat[][Example specific caption %\label{fig:example_label}
	]{
	\resizebox{.45\textwidth}{!}{
   \includegraphics{file_name} % da modificare
	}}\qquad
	\subfloat[][Example specific caption %\label{fig:example_label}
	]{
	\resizebox{.45\textwidth}{!}{
   \includegraphics{file_name} % da modificare
	}}
	\caption{Example global caption} 
	%\label{fig:example_global_label}
\end{figure}




% Per affiancare DUE FIGURE con caption distinte
\begin{figure}[htbp]
\begin{floatrow}
   \centering
\ffigbox{%
\resizebox{.45\textwidth}{!}{ 
   \includegraphics[width=.45\textwidth]{file_name} % da modificare
   }}{%
    \caption{Example caption} % da modificare
%    \label{fig:example_label}
}\quad
\ffigbox{%
\resizebox{.45\textwidth}{!}{
   \includegraphics[width=.6\textwidth]{file_name} % da modificare
   }}{%
   \caption{Example caption} % da modificare
%   \label{fig:example_label}
   }
\end{floatrow}
\end{figure}